\chapter{Planteamiento del Problema}
\section{Descripción de la Realidad Problemática}

La industria cosmética y de cuidado de la piel ha visto un crecimiento exponencial en las últimas décadas. Según datos de Statista, el mercado global de productos para el cuidado de la piel superó los 130 mil millones de dólares en 2023, con una tasa de crecimiento anual compuesta de aproximadamente el 0,045 \cite{statista2023}. Este auge refleja la creciente demanda por soluciones que permitan a los consumidores mejorar su apariencia, retardar los signos del envejecimiento y solucionar problemas estéticos como las arrugas, los poros dilatados y las manchas faciales.

A pesar de la variedad de productos y tratamientos disponibles en el mercado, persisten varios desafíos relacionados con la evaluación precisa y personalizada de los problemas de la piel. Uno de los mayores obstáculos en la actualidad es la falta de herramientas tecnológicas que permitan realizar un análisis profundo y cuantitativo de las características morfológicas de la piel. Problemas como las arrugas, los poros dilatados y las manchas son difíciles de evaluar de manera objetiva, ya que la mayoría de los diagnósticos aún dependen de la observación manual o de sistemas de análisis que no logran captar las sutilezas y complejidades de la piel facial \cite{phillips2020}.

La precisión en la detección de estas características es crucial para el desarrollo de tratamientos más efectivos. Por ejemplo, la identificación temprana de las arrugas incipientes permitiría la aplicación de productos antiarrugas de manera preventiva, antes de que las líneas se profundicen. Sin embargo, las evaluaciones actuales a menudo son subjetivas y pueden variar según el profesional o las herramientas utilizadas. En este sentido, estudios recientes han demostrado que las redes neuronales convolucionales (CNN) tienen el potencial de revolucionar la manera en que se realiza la segmentación y análisis de imágenes de piel \cite{esteva2017}. Estas redes permiten procesar grandes volúmenes de datos visuales y detectar patrones morfológicos con un nivel de precisión que supera las técnicas tradicionales.

Otro desafío significativo es la evaluación de poros y manchas. Los poros dilatados son una preocupación estética común, especialmente entre personas con piel grasa. La falta de herramientas capaces de cuantificar y analizar adecuadamente el tamaño y la densidad de los poros limita las recomendaciones de tratamiento personalizadas \cite{jia2019}. De manera similar, las manchas faciales, que pueden aparecer debido a la edad, la exposición solar o factores hormonales, requieren una evaluación temprana para evitar su progresión. Las CNN, al especializarse en la segmentación de imágenes, pueden proporcionar una solución eficaz para mapear la distribución y evolución de estas imperfecciones cutáneas \cite{esteva2017}.

Estadísticamente, se ha encontrado que más del 0,7 de las personas mayores de 25 años muestran algún signo de envejecimiento facial, como arrugas y manchas, lo que impulsa la demanda de productos antiarrugas y despigmentantes \cite{aad2022}. Sin embargo, el éxito de estos productos depende en gran medida de una evaluación precisa del estado de la piel. Actualmente, los diagnósticos imprecisos o subjetivos pueden llevar a la aplicación de productos inapropiados, lo que no solo afecta la satisfacción del consumidor, sino que también reduce la efectividad de los tratamientos \cite{khatri2018}.

Este contexto evidencia la necesidad urgente de desarrollar tecnologías avanzadas que mejoren la precisión en la evaluación estética de la piel. Las redes neuronales convolucionales ofrecen una herramienta prometedora para abordar este problema, al permitir una segmentación detallada de las características morfológicas clave de la piel, como arrugas, poros y manchas. El uso de estas tecnologías no solo permitiría mejorar los diagnósticos estéticos, sino que también contribuiría a la creación de tratamientos personalizados más efectivos, incrementando la satisfacción del usuario final.



\section{Formulación del Problema}

\subsection{Problema General}
PG: \newcommand{\ProblemaGeneral}{

¿Cómo afecta la falta de un sistema de segmentación  de características morfológicas de la piel facial en la detección de arrugas, poros y manchas?
}
\ProblemaGeneral
\subsection{Problemas Específicos}
\newcommand{\Pbone}{
¿Cómo se medirá la eficiencia del sistema de segmentación morfológica en la detección de arrugas, poros y manchas?
}
\newcommand{\Pbtwo}{
¿Cómo se desarrollará el sistema de segmentación basado en redes neuronales convolucionales (CNN)?
}
\newcommand{\Pbthree}{
¿De dónde se obtendrá la data para entrenar y validar el sistema de segmentación?
}

\newcommand{\Pbfour}{
¿Cómo se determinará cuál es el mejor modelo de segmentación para la detección de características morfológicas?
}

\begin{itemize}
	\item PE1: {\Pbone}
	\item PE2: {\Pbtwo}
	\item PE3: {\Pbthree}
	\item PE4: {\Pbfour}
\end{itemize}

\section{Objetivos de la Investigación}
A continuación, se presentan el objetivo general y los objetivos específicos.
\subsection{Objetivo General}
OG: \newcommand{\ObjetivoGeneral}{
	
Desarrollar un sistema avanzado de segmentación de características morfológicas en imágenes de piel facial, centrado en arrugas, poros y manchas, utilizando redes neuronales convolucionales para mejorar la precisión en la evaluación estética y la personalización de tratamientos cosméticos.

}
\ObjetivoGeneral
\subsection{Objetivos Específicos}
\newcommand{\Objone}{
Desarrollar métricas de evaluación como precisión, recall, F1-score y AUC-ROC para medir la eficiencia del sistema de segmentación en la identificación de características morfológicas de la piel facial.
}

\newcommand{\Objtwo}{
Desarrollar e implementar un sistema de segmentación utilizando redes neuronales convolucionales, adaptando sus arquitecturas para la detección y diferenciación de características morfológicas de la piel como arrugas, poros y manchas.
}

\newcommand{\Objthree}{
Recopilar y preparar un conjunto de datos de imágenes faciales, asegurando que contenga suficiente variedad en términos de diferentes tipos de piel y problemas cutáneos (arrugas, poros y manchas), para entrenar y validar el sistema de segmentación.
}
\newcommand{\Objfour}{
Comparar diferentes arquitecturas de redes neuronales convolucionales y técnicas de aprendizaje profundo, evaluando su desempeño en la segmentación de arrugas, poros y manchas, para seleccionar el modelo que ofrezca el mejor equilibrio entre precisión y eficiencia computacional.
}

\begin{itemize}
	\item OE1: {\Objone}
	\item OE2: {\Objtwo}
	\item OE3: {\Objthree}
	\item OE4: {\Objfour}
\end{itemize}



\section{Justificación de la Investigación}

\subsection{Teórica}
Este estudio formula el desarrollo de un modelo de redes neuronales convolucionales (CNN), que han demostrado ser altamente eficaces para tareas de segmentación y clasificación de imágenes. El desarrollo de un sistema de segmentación morfológica permitirá avanzar en el campo de la dermatología computacional y los sistemas inteligentes aplicados a la cosmética y el cuidado de la piel. Además, se contribuirá a la literatura sobre la intersección entre las ciencias de la salud y la inteligencia artificial, explorando el uso de técnicas como el deep learning para detectar y segmentar características faciales como arrugas, poros y manchas.

\subsection{Práctica}
El sistema de segmentación propuesto puede tener aplicaciones directas en la industria cosmética, dermatológica y médica. La detección temprana y precisa de imperfecciones en la piel es esencial para tratamientos preventivos y correctivos. Este sistema permitirá automatizar y optimizar la evaluación cutánea, mejorando la precisión y reduciendo el tiempo necesario para diagnósticos o recomendaciones personalizadas de tratamiento. Esto también podría ser útil para mejorar la personalización de productos de belleza, apoyando la creación de soluciones adaptadas a las necesidades específicas de la piel de cada persona.

\subsection{Metodológica}
Utilizar arquitecturas probadas de CNN para segmentar características específicas como arrugas, poros y manchas proporciona una metodología efectiva y replicable. Al emplear técnicas de deep learning, se puede mejorar la precisión en la segmentación de imágenes de la piel facial, lo que facilitará comparaciones entre distintas aproximaciones y modelos. La experimentación con distintos modelos también ayudará a determinar cuál es el enfoque más eficiente y preciso para este tipo de segmentación.

\section{Delimitación del Estudio}
Este estudio se centrará exclusivamente en la segmentación de arrugas, poros y manchas en imágenes de piel facial. No se abordarán otras imperfecciones cutáneas ni se desarrollarán sistemas de diagnóstico clínico. El enfoque está dirigido hacia el sector cosmético y de belleza, en lugar de la dermatología médica. Además, el estudio se delimitará a imágenes en dos dimensiones (2D) y no incluirá análisis en 3D.

\subsection{Espacial}
El estudio utilizará imágenes faciales provenientes de bases de datos públicas con imágenes de alta calidad. Se concentrará en muestras representativas de diferentes tipos de piel, obtenidas principalmente de poblaciones de distintas regiones geográficas para asegurar la diversidad en la segmentación de las características faciales.
  
\subsection{Temporal}
El desarrollo del estudio y la validación del sistema se llevarán a cabo en un periodo de aproximadamente 6 a 12 meses, incluyendo fases de recolección de datos, desarrollo del modelo, pruebas y evaluación del sistema. Las imágenes utilizadas para el entrenamiento y pruebas corresponderán a muestras recolectadas en los últimos cinco años para asegurar la actualidad de las características cutáneas.

\subsection{Conceptual}
Este estudio aborda conceptos fundamentales de la segmentación de imágenes, redes neuronales convolucionales, características morfológicas de la piel y técnicas de procesamiento de imágenes. Conceptos clave como segmentación, CNN, arrugas, poros y manchas se definirán claramente para establecer el marco teórico y práctico del trabajo. Además, se discutirán términos como detección automática y análisis morfológico en el contexto del cuidado de la piel.