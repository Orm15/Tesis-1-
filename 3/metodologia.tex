\chapter{Metodología de la Investigación}
\section{Diseño de la investigación}

El diseño de esta investigación es de tipo experimental, ya que implica la creación y evaluación de un sistema de segmentación utilizando CNN para identificar características morfológicas de la piel. Se llevarán a cabo diferentes experimentos con arquitecturas CNN para determinar la que ofrezca el mejor equilibrio entre precisión y eficiencia computacional \cite{esteva2017, khatri2018}. Este diseño permite realizar pruebas controladas sobre el conjunto de datos de imágenes faciales para comparar el desempeño de diferentes modelos y seleccionar el que mejor se adapte a las necesidades del análisis estético.


\subsection{Alcance de la investigación}
El alcance de esta investigación se enfoca exclusivamente en la segmentación de arrugas, poros y manchas en imágenes faciales en dos dimensiones (2D). Este trabajo está orientado al sector cosmético y de belleza, en lugar de la dermatología médica, y no abordará el diagnóstico de otras condiciones cutáneas. Las imágenes provendrán de bases de datos públicas y se seleccionarán muestras representativas de diferentes tipos de piel y regiones geográficas para asegurar diversidad en los datos. La temporalidad del estudio abarca un período de aproximadamente 6 a 12 meses, desde la recolección de datos hasta la validación del modelo \cite{statista2023, aad2022}.

\subsection{Enfoque de la investigación}
El enfoque de esta investigación es cuantitativo, ya que se busca desarrollar un sistema de segmentación para detectar características morfológicas de la piel facial mediante redes neuronales convolucionales (CNN) y analizar su efectividad con métricas cuantitativas, como precisión, recall, F1-score y AUC-ROC. Este enfoque permitirá evaluar el desempeño del sistema en la detección de arrugas, poros y manchas, proporcionando resultados medibles y objetivos \cite{esteva2017, jia2019}. Al emplear técnicas de aprendizaje profundo, el estudio pretende optimizar la precisión en la segmentación de características faciales, aplicando un marco metodológico replicable y sistemático \cite{phillips2020}.

La presente investigación tiene un enfoque cuantitativo, esto dado que la variable independiente usa valores numéricos y/o estadísticos para su medición. Los resultados de la variable de Deep Learning deberán ser medidos a través de valores numéricos y, en mayor medida, estadísticos.

\subsection{Población}
La población de este estudio se compone de imágenes faciales representativas de personas con diversas edades, géneros y tipos de piel. Específicamente, estas imágenes muestran características morfológicas que se asocian con arrugas, poros dilatados y manchas de la piel facial y se han obtenido públicas y especializadas en dermatología y cosméticos bases de datos. Debido a la orientación del enfoque en el problemas estéticos, la población abarcaba imágenes de piel con claras imperfecciones y piel sin e incidencias asignadas. Así, el alcance de la población se determina como diverso y completo, asegurando la inclusión de imágenes que representa una amplia gama de condiciones de la piel. Finalmente, resulta esencial agregar diversidad geográfica, ya que ciertas diferencias geográficas.
\subsection{Muestra}
La muestra de la investigación comprenderá una parte de aproximadamente 5000 retratos faciales seleccionados de archivos públicos y privados Estas imágenes se seleccionarán mediante muestreo basado en estratos, lo que garantizará una representación uniforme en diferentes categorías de edad, identidades masculinas y femeninas y pigmentación dérmica variable La lista también tendrá imágenes con diferentes tamaños y claridad, mostrando varios tipos de arrugas, poros y manchas, asegurándose de que el grupo muestre condiciones reales de la piel Los usaremos para enseñar, verificar y desafiar nuestro modelo de visión por computadora, asegurándonos de que funcione bien en la vida real con mucha variedad.
\section{Metodología de Implementación de la Solución}
%%%%%%%%%%%%%%%%%%%
\afterpage{%
\begin{landscape}
    \clearpage  % Asegura que el contenido empiece en una nueva página
    \section{Diagrama de la Metodología}
    Se propuso esta Metodología. Este se presenta en la Figura \ref{3:fig303}.

    \begin{figure}[!h]
        \begin{center}
            \includegraphics[width=\textwidth]{3/figures/metodologia.png}
            \caption[Cronograma de actividades]{Cronograma de actividades.\\
                Fuente: Elaboración propia.}
            \label{3:fig303}
        \end{center}
    \end{figure}
    \clearpage  % Evita que el contenido siguiente interfiera
\end{landscape}}




%%%%%%%%%%%%%%%%%%%%
La implementación de la solución propuesta se llevará a cabo en varias etapas clave, que garantizarán la preparación adecuada de los datos, el desarrollo de modelos efectivos y su evaluación rigurosa para abordar las necesidades específicas de segmentación de características morfológicas de la piel facial.

\subsection{Adquisición y Preparación de los Datos}
Se recopilará un conjunto de datos compuesto por imágenes faciales provenientes de bases de datos públicas y privadas. Se priorizarán imágenes que reflejen diversidad en tipos de piel, condiciones morfológicas (como arrugas, poros dilatados y manchas) y niveles de calidad visual. Cada imagen será etiquetada manualmente para clasificar las características específicas en categorías relevantes, como "arrugas incipientes", "poros dilatados" o "manchas hiperpigmentadas".

Durante esta etapa, se planea realizar las siguientes acciones:
\begin{itemize}
    \item \textbf{Exploración inicial:} Se analizará el conjunto de datos para identificar inconsistencias, como imágenes corruptas, valores faltantes o clases desbalanceadas.
    \item \textbf{Limpieza:} Se eliminarán imágenes no útiles (mal etiquetadas o de baja resolución) y se corregirán anomalías en las etiquetas.
    \item \textbf{Aumento de Datos:} Se aplicarán técnicas como rotación, escalado, cambios de iluminación y adición de ruido para abordar problemas de desbalanceo y mejorar la capacidad de generalización de los modelos.
    \item \textbf{Preprocesamiento:} Las imágenes se redimensionarán a una resolución uniforme, se normalizarán los valores de color y se convertirán a un formato estándar que facilite su uso en redes neuronales convolucionales.
\end{itemize}

\subsection{Desarrollo de los Modelos de Segmentación}
Se desarrollarán modelos de segmentación de características morfológicas de la piel utilizando Redes Neuronales Convolucionales (CNN), dada su eficacia comprobada en tareas de análisis de imágenes. Las arquitecturas que se implementarán incluyen:

\begin{itemize}
    \item \textbf{U-Net:} Se utilizará por su capacidad para realizar segmentaciones precisas en áreas específicas, como arrugas y poros, gracias a su diseño simétrico que facilita la reconstrucción de imágenes segmentadas \cite{ronneberger2015}.
    \item \textbf{ResNet:} Se probará para capturar características profundas y complejas mediante su estructura de aprendizaje residual, mejorando el rendimiento en imágenes con texturas complejas \cite{he2016}.
    \item \textbf{Modelos híbridos con Vision Transformer (ViT):} Se explorará su uso para la segmentación de patrones más sutiles, aprovechando su enfoque basado en atención combinado con la extracción de características de las CNN \cite{dosovitskiy2020}.
\end{itemize}

El entrenamiento de los modelos se realizará aplicando las siguientes técnicas:
\begin{itemize}
    \item \textbf{Optimización con Adam:} Para garantizar convergencia rápida y eficiente.
    \item \textbf{Data augmentation:} Se emplearán técnicas de aumento de datos para mejorar la robustez del modelo ante variaciones en el conjunto de datos.
    \item \textbf{Validación cruzada:} El conjunto de datos se dividirá en pliegues para evaluar el desempeño del modelo en diferentes particiones.
\end{itemize}

\subsection{Evaluación y Validación del Sistema}
El desempeño del sistema de segmentación será evaluado utilizando métricas específicas para problemas de clasificación y segmentación:
\begin{itemize}
    \item \textbf{Precisión (Accuracy):} Proporción de predicciones correctas sobre el total de predicciones.
    \item \textbf{Recall (Sensibilidad):} Proporción de características positivas correctamente identificadas.
    \item \textbf{F1-Score:} Promedio armónico de precisión y recall, útil para conjuntos de datos desbalanceados.
    \item \textbf{AUC-ROC:} Área bajo la curva ROC para medir la capacidad del modelo de distinguir entre clases.
\end{itemize}

Adicionalmente, se utilizará una matriz de confusión para analizar en detalle los errores cometidos por los modelos, lo que permitirá identificar patrones en los casos mal clasificados.

\subsection{Implementación y Pruebas Finales}
El modelo seleccionado será integrado en un prototipo funcional que permitirá procesar imágenes faciales y generar un reporte detallado de las características segmentadas. Este prototipo incluirá una interfaz amigable que mostrará:
\begin{itemize}
    \item Mapas de calor que destacarán áreas con arrugas, poros dilatados y manchas.
    \item Gráficos comparativos que mostrarán la evolución de las características morfológicas.
    \item Recomendaciones personalizadas basadas en el análisis segmentado.
\end{itemize}

Las pruebas finales se realizarán en un entorno simulado para verificar su rendimiento en casos prácticos, como evaluaciones estéticas en clínicas dermatológicas o la personalización de tratamientos cosméticos.

\section{Metodología para la Medición de Resultados de la Implementación}

Para garantizar la correcta evaluación del sistema, se emplearán las siguientes métricas basadas en la matriz de confusión:
\begin{itemize}
    \item \textbf{Precisión (Accuracy):} \[ accuracy = \frac{TP + TN}{TP + TN + FP + FN} \]
    \item \textbf{Recall:} \[ recall = \frac{TP}{TP + FN} \]
    \item \textbf{Precisión Positiva:} \[ precision = \frac{TP}{TP + FP} \]
    \item \textbf{F1-Score:} \[ F1 = 2 \cdot \frac{precision \cdot recall}{precision + recall} \]
\end{itemize}

Cada métrica será calculada para determinar la efectividad del modelo en la segmentación de arrugas, poros y manchas, evaluando su capacidad de proporcionar predicciones precisas y consistentes en datos no vistos.
%%%%%%%%%%%%%%%%%%%%%%%%%%%%%%%%%%%%%%

\begin{landscape}
	\section{Cronograma de actividades y presupuesto}
	Se propuso un cronograma para la investigación. Conforma desde el inicio hasta ser terminada con la sustentación final planeada para mediados del año 2024. Este se presneta en la Figura \ref{3:fig303}.

	\begin{figure}[!ht]
		\begin{center}
			\includegraphics[width=1.50\textwidth]{3/figures/gant.png}
			\caption[Cronograma de actividades]{Cronograma de actividades.\\
				Fuente: Elaboración propia.}
			\label{3:fig303}
		\end{center}
	\end{figure}
	
\end{landscape}
%%%%%%%%%%%%%%%%%%%%%%%%%%%%%%%%%%%%%%%%

%%%%%%%%%%%%%%%%%%%%%%%%%%%%%%%




Se determinó el siguiente presupuesto necesario para la elaboración completa de la investigación. Este se presenta en la Tabla \ref{3:table1}.

\begin{table}[H]
	\caption[Presupuesto]{Presupuesto estimado para el desarrollo del sistema de segmentación morfológica.}
	\label{3:table1}
	\centering
	\small
	\begin{tabular}{llll}
		\specialrule{.1em}{.05em}{.05em}
		{Grupo} & {Item} & {Costo (soles)} & {Subtotal} \\ 
		\specialrule{.1em}{.05em}{.05em}
		\multirow{3}{4cm}{Recursos materiales} 
		& {Laptop de alto rendimiento} & {S/ 7,500.00} & {} \\ 
		& {Materiales de escritorio} & {S/ 150.00} & {} \\
		& {Dispositivo de almacenamiento externo} & {S/ 300.00} & {S/ 7,950.00} \\ 
		\cline{1-4}
		\multirow{3}{4cm}{Software y servicios} 
		& {Licencia de software (Python/IDE)} & {S/ 50.00} & {} \\
		& {Renta de servidor en la nube} & {S/ 500.00} & {} \\
		& {Acceso a bases de datos de imágenes} & {S/ 300.00} & {S/ 850.00} \\ 
		\cline{1-4}
		\multirow{3}{4cm}{Costos académicos} 
		& {Matrícula en Trabajo de Tesis II} & {S/ 375.00} & {} \\
		& {Cuotas de Trabajo de Tesis II} & {S/ 1,044.00} & {S/ 1,419.00} \\ 
		\cline{1-4}
		\multirow{2}{4cm}{Extras} 
		& {Consultorías especializadas} & {S/ 200.00} & {} \\
		& {Movilidad y transporte} & {S/ 300.00} & {S/ 500.00} \\ 
		\specialrule{.1em}{.05em}{.05em} 
		{} & {Total} & {} & {S/ 10,719.00} \\ 
		\specialrule{.1em}{.05em}{.05em}
	\end{tabular}
	\begin{flushleft}	
		\small Fuente: Elaboración propia.
	\end{flushleft}
\end{table}




