\subsection{Hipótesis General}
HG: \newcommand{\HipotesisGeneral}{
	El desarrollo de un Sistema Avanzado de Segmentación de Características Morfológicas Faciales utilizando Redes Neuronales Convolucionales mejorará significativamente la precisión en la detección de Arrugas y Manchas}
\HipotesisGeneral


\subsection{Hipótesis Específicas}
\newcommand{\Hone}{
La utilización de un conjunto de datos diverso de imágenes faciales, que contemple características morfológicas faciales, como arrugas y manchas, incrementará la capacidad del sistema para generalizar y segmentar con precisión las características cutáneas.
}
\newcommand{\Htwo}{
Un sistema de segmentación basado en Redes Neuronales Convolucionales permitirá una detección más precisa y diferenciada de las características morfológicas faciales, tales como arrugas y manchas, superando en rendimiento a los métodos tradicionales de segmentación.
}
\newcommand{\Hthree}{
La utilización de métricas de evaluación cuantitativas, junto con una comparación sistemática entre distintas arquitecturas de Redes Neuronales Convolucionales —incluyendo aquellas con mecanismos de atención—, permitirá una valoración objetiva, confiable y robusta del desempeño del sistema de segmentación, así como la identificación del modelo más eficiente en términos de precisión y rendimiento computacional en la detección de arrugas y manchas faciales.
}
\newcommand{\Hfour}{
La implementación de un sistema de segmentación de características morfológicas en tiempo real, utilizando redes neuronales convolucionales potenciadas con mecanismos de atención, permitirá el procesamiento eficaz de imágenes faciales desde video en vivo, mejorando la precisión en entornos dinámicos.
}

\begin{itemize}
	\item HE1: {\Hone}
	\item HE2: {\Htwo}
	\item HE3: {\Hthree}
	\item HE4: {\Hfour}
\end{itemize}