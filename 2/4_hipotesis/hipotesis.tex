\subsection{Hipótesis General}
HG: \newcommand{\HipotesisGeneral}{
	El desarrollo de un sistema avanzado de segmentación de características morfológicas de la piel facial utilizando redes neuronales convolucionales (CNN) mejorará significativamente la precisión en la detección de arrugas, poros y manchas, lo que permitirá la personalización de tratamientos cosméticos y estéticos.
}
\HipotesisGeneral


\subsection{Hipótesis Específicas}
\newcommand{\Hone}{

El uso de métricas de evaluación como precisión, recall, F1-score y AUC-ROC permitirá medir con mayor objetividad la eficiencia del sistema de segmentación en la identificación de arrugas, poros y manchas faciales.
}
\newcommand{\Htwo}{
Un sistema de segmentación basado en redes neuronales convolucionales (CNN) logrará una mejor detección y diferenciación de características morfológicas de la piel en comparación con métodos tradicionales de análisis estético.
}
\newcommand{\Hthree}{
La incorporación de un conjunto de datos diverso de imágenes faciales mejorará la capacidad del sistema de segmentación para generalizar en distintos tipos de piel y problemas cutáneos.
}
\newcommand{\Hfour}{
La comparación entre distintas arquitecturas de redes neuronales convolucionales permitirá identificar un modelo óptimo que ofrezca el mejor balance entre precisión en la segmentación y eficiencia computacional.
}

\begin{itemize}
	\item HE1: {\Hone}
	\item HE2: {\Htwo}
	\item HE3: {\Hthree}
	\item HE4: {\Hfour}
\end{itemize}