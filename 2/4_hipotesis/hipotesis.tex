\subsection{Hipótesis General}
HG: \newcommand{\HipotesisGeneral}{
	El desarrollo de un sistema avanzado de segmentación de características morfológicas de la piel facial utilizando redes neuronales convolucionales (CNN), con mecanismos de atención, mejorará significativamente la precisión en la detección de arrugas y manchas, permitiendo una segmentación más precisa y eficiente en tiempo real.}
\HipotesisGeneral


\subsection{Hipótesis Específicas}
\newcommand{\Hone}{
La utilización de un conjunto de datos diverso de imágenes faciales, que contemple distintos tipos de piel y características morfológicas como arrugas y manchas, incrementará la capacidad del sistema para generalizar y segmentar con precisión las características cutáneas.
}
\newcommand{\Htwo}{
Un sistema de segmentación basado en Redes Neuronales Convolucionales permitirá una detección más precisa y diferenciada de las características morfológicas de la piel, tales como arrugas y manchas, superando en rendimiento a los métodos tradicionales de segmentación.
}
\newcommand{\Hthree}{
El uso de métricas de evaluación cuantitativas proporcionará una valoración objetiva, confiable y robusta del desempeño del sistema de segmentación en la detección de arrugas y manchas faciales.
}
\newcommand{\Hfour}{
La comparación sistemática entre distintas arquitecturas de Redes Neuronales Convolucionales, incluyendo aquellas con mecanismos de atención, permitirá identificar el modelo más eficiente en términos de precisión y rendimiento computacional para la segmentación de características morfológicas cutáneas.
}
\newcommand{\Hfive}{
La implementación de un sistema de segmentación de características morfológicas en tiempo real, utilizando redes neuronales convolucionales potenciadas con mecanismos de atención, permitirá el procesamiento eficaz de imágenes faciales desde video en vivo, mejorando la precisión en entornos dinámicos.
}

\begin{itemize}
	\item HE1: {\Hone}
	\item HE2: {\Htwo}
	\item HE3: {\Hthree}
	\item HE4: {\Hfour}
	\item HE5: {\Hfive}
\end{itemize}