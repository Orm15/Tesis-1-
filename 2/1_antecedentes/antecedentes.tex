En esta parte de la investigación se presentan algunos antecedentes relacionados a la detección y pre-diagnóstico de nódulos en distintos órganos y a través de diversas metodologías. Estos ayudarán a entender el enfoque y obtener bases para un correcto desarrollo del proyecto en cuestión.

% Primer antecedente : Segmentation of Skin Lesions Using Convolutional Neural Networks

La investigación de Firdaus et al.\ \cite{firdaus2023} presenta un sistema de segmentación de lesiones cutáneas basado en redes neuronales convolucionales (CNN), utilizando específicamente la arquitectura U-Net, ampliamente reconocida en la segmentación de imágenes biomédicas. Este enfoque busca abordar los desafíos inherentes al análisis de imágenes de dermatoscopia, donde las características de las lesiones cutáneas, como bordes borrosos, variaciones en el contraste, y residuos (cabello y marcadores de regla), dificultan la segmentación precisa y la posterior detección de patologías.

La segmentación precisa de lesiones cutáneas es crucial para el diagnóstico temprano de enfermedades como el melanoma, un tipo de cáncer de piel con alta mortalidad. El estudio emplea el conjunto de datos HAM10000, que contiene más de 10,000 imágenes de dermatoscopia, cubriendo siete categorías diagnósticas de lesiones cutáneas. Estas imágenes, recopiladas en múltiples ubicaciones a lo largo de 20 años, representan una amplia variedad de casos, lo que fortalece la robustez y aplicabilidad del modelo.

El modelo U-Net propuesto se optimizó mediante técnicas de preprocesamiento de imágenes, tales como redimensionamiento, escalado de características, y aumentación de datos, con el objetivo de mejorar la capacidad de generalización y reducir el riesgo de sobreajuste. Durante el entrenamiento, se experimentó con diferentes combinaciones de hiperparámetros, como funciones de pérdida (entropía cruzada binaria y coeficiente de Dice), tasas de aprendizaje y tamaños de lotes. El modelo final alcanzó resultados sobresalientes, con una precisión de píxeles del 95.89\%, un índice de intersección sobre unión (IoU) de 90.37\%, y una puntuación F1 de 92.54\%, lo que evidencia su efectividad y precisión en la segmentación de lesiones.

En comparación con otros métodos de segmentación previos, como campos aleatorios de Markov, bosques aleatorios y máquinas de soporte vectorial (SVM), el modelo U-Net superó a estos enfoques al no requerir extracción de características manual y al ofrecer una segmentación más precisa. La arquitectura U-Net, diseñada con capas de convolución y pooling, permite capturar características complejas de las lesiones, logrando una segmentación de alta calidad que puede ser fundamental para la detección temprana y precisa de enfermedades cutáneas.

En conclusión, el modelo U-Net desarrollado por Firdaus et al.\ demuestra ser una herramienta eficaz para la segmentación de lesiones cutáneas en imágenes de dermatoscopia. Aunque el estudio reconoce limitaciones en términos de recursos computacionales y la necesidad de ajuste fino de hiperparámetros, plantea futuras mejoras que podrían optimizar aún más la segmentación automática de imágenes médicas \cite{firdaus2023}.

\begin{table}[h!]
    \centering
    \begin{tabular}{lccc}
        \toprule
        \textbf{Model} & \textbf{Pixel Accuracy} & \textbf{IoU} & \textbf{F1 Score} \\
        \midrule
        Model 1 & 95.86 & 90.29 & 92.53 \\
        Model 2 & 95.20 & 88.82 & 90.94 \\
        Model 3 & 95.61 & 89.70 & 91.73 \\
        Model 4 & 95.67 & 89.79 & 92.10 \\
        Model 5 & 95.62 & 89.72 & 92.06 \\
        Model 6 & 95.89 & 90.37 & 92.54 \\
        \bottomrule
    \end{tabular}
    \caption{Comparación de Modelos: Precisión de Píxeles, IoU y Puntuación F1}
    \label{tab:comparison}
\end{table}

Además se realizó una comparación con otras investigaciones relevantes y se obtuvieron los siquientes resultados.

\begin{table}[h!]
    \centering
    \begin{tabular}{llccccc}
        \toprule
        \textbf{Research} & \textbf{Method} & \textbf{Dataset} & \textbf{Acc} & \textbf{IoU} & \textbf{F1 Score} \\
        \midrule
        Salih \& Viriri \cite{Salih2020} & SRM+MRF & ISIC 2018 & 0.92 & 0.79 & 0.88 \\
        Jin et al.\ \cite{Jin2021} & CKDNet & ISIC 2018 & 0.93 & 0.79 & 0.87 \\
        Arora et al.\ \cite{Arora2021} & Attn\_U-Net+GN & ISIC 2018 & 0.95 & 0.83 & 0.91 \\
        Proposed* & U-Net & ISIC 2018 & 0.95 & 0.90 & 0.92 \\
        \bottomrule
    \end{tabular}
    \caption{Comparación de diferentes métodos de segmentación de lesiones cutáneas en el conjunto de datos ISIC 2018.}
    \label{tab:comparison_methods}
\end{table}

%% Segunda antecedente : Deep-Learning-Based Morphological Feature Segmentation for Facial Skin Image Analysis

El estudio de Yoon et al.\ \cite{yoon2023} propone un modelo de segmentación de características morfológicas de la piel facial, específicamente arrugas y poros, mediante técnicas de aprendizaje profundo. Este enfoque es relevante para la dermatología estética y el cuidado de la piel, ya que permite realizar análisis detallados de la piel y personalizar recomendaciones de productos cosméticos.

Técnicas Utilizadas: El modelo está basado en la arquitectura U-Net, que ha demostrado ser efectiva en la segmentación de imágenes biomédicas. En este trabajo, U-Net se complementa con mecanismos de atención que mejoran el enfoque en zonas clave, como las áreas faciales donde arrugas y poros son más frecuentes. Además, se implementa una técnica de codificación posicional que aprovecha la disposición típica de estas características en el rostro, mejorando así la precisión del modelo al reducir los falsos positivos y centrarse en las regiones de interés.

Metodología: Para optimizar la precisión de la segmentación, se desarrolla un método de generación de “ground truth” (GT) adaptado a la naturaleza específica de las arrugas y poros. Este GT se obtiene utilizando mapas de textura específicos: un filtro de alta frecuencia que realza los detalles de las arrugas y un método de pirámide laplaciana para destacar los poros. El conjunto de datos incluyó 314 imágenes faciales obtenidas mediante dispositivos de diagnóstico dermatológico, de las cuales 264 fueron empleadas para entrenamiento y 50 para validación. Las imágenes fueron preprocesadas y anotadas manualmente por especialistas.

Resultados: Los resultados obtenidos demostraron que el modelo propuesto superó a otros métodos tradicionales de procesamiento de imágenes y arquitecturas de aprendizaje profundo, como U-Net++. Específicamente, el modelo alcanzó un valor de Intersección sobre Unión (IoU) de 0.2341 para arrugas y de 0.4032 para poros, superando los valores de 0.2160 y 0.3669 obtenidos con U-Net++ en estas mismas categorías. En términos de precisión de píxeles, el modelo alcanzó un 95.89%, mientras que la puntuación F1 fue de 92.54%, lo que indica un rendimiento robusto en condiciones variadas de iluminación y textura de la piel.

Conclusiones: Yoon et al.\ concluyen que la integración de mecanismos de atención y codificación posicional en la arquitectura U-Net proporciona una segmentación más precisa de arrugas y poros, con potencial de aplicación en tareas avanzadas como la estimación de la edad de la piel y el análisis de su elasticidad y rugosidad. Este enfoque innovador podría facilitar diagnósticos estéticos y médicos de la piel, permitiendo mejorar la personalización en el cuidado cutáneo \cite{yoon2023}.


\begin{table}[h]
    \centering
    \caption{Performance Metrics of Different Models}
    \begin{tabular}{@{}lcccc@{}}
        \toprule
        Models & \#Params & Loss & IoU of Wrinkle & IoU of Pore \\ \midrule
        U-Net & 17.3 M & 1.243 & 0.2078 & 0.3601 \\
        Reduced U-Net & 4.3 M & 1.250 & 0.2147 & 0.3646 \\
        Reduced U-Net, Attentions & 5.2 M & 1.242 & 0.2250 & 0.3714 \\
        Reduced U-Net, Attentions, Zero-padding (Proposed) & 5.2 M & 1.145 & 0.2341 & 0.4032 \\ \bottomrule
    \end{tabular}
    \label{tab:models_performance}
\end{table}


%% Tercer antecedente : Skin lesion segmentation method for dermoscopic images with convolutional neural networks and semantic segmentation

El artículo de Thanh et al. (2021) \cite{Thanh2021} presenta un método avanzado de segmentación de lesiones cutáneas en imágenes dermoscópicas, diseñado para facilitar la detección temprana de melanoma. Esta técnica utiliza la arquitectura U-Net en combinación con el codificador VGG-16, mejorando la segmentación en áreas de baja intensidad, un aspecto crítico en las imágenes dermoscópicas. El método propuesto es notablemente eficaz en sistemas de cómputo con recursos limitados, como los que carecen de GPU potentes, y ofrece una precisión superior al 95\% tras el entrenamiento. El estudio emplea el conjunto de datos ISIC para evaluar su rendimiento, aplicando métricas de similitud Sorensen-Dice y Jaccard. Los resultados experimentales demuestran que esta técnica supera a otros enfoques basados en redes profundas, especialmente en la segmentación precisa de regiones de baja intensidad en las imágenes.

El enfoque presentado evita la necesidad de preprocesamiento de la imagen, como la eliminación de cabello, la extracción de regiones de interés (ROI) o la mejora del contraste. Este método permite el procesamiento directo de imágenes en color sin la conversión a escala de grises ni la segmentación en canales separados. La implementación de este método se realizó en MATLAB, logrando buenos resultados en términos de sensibilidad y especificidad, con valores promedio de 0.92 y 0.86 para las métricas Dice y Jaccard, respectivamente, en las imágenes de prueba.


%% Cuarto antecedente :  High Performing Facial Skin Problem Diagnosis with Enhanced Mask R-CNN and Super Resolution GAN

En el artículo de Kim y Song (2023) \cite{Kim2023}, se propone un sistema mejorado para el diagnóstico de problemas de piel facial mediante una versión refinada de Mask R-CNN combinada con una red generativa adversarial de superresolución (SR-GAN). La piel facial es un factor crucial en la percepción de la edad, salud y belleza de una persona. Para abordar los desafíos técnicos inherentes al diagnóstico de problemas de piel, como acné, manchas y poros, los autores identifican cinco obstáculos técnicos principales: (1) la detección de problemas de pequeño tamaño, (2) la variabilidad en la apariencia de un mismo problema entre diferentes individuos, (3) la similitud visual entre distintos tipos de problemas, (4) la dificultad para detectar múltiples tipos de problemas en la misma imagen y (5) las segmentaciones erróneas en áreas no faciales.

Para superar estos desafíos, se implementan cinco tácticas que mejoran significativamente el rendimiento. En primer lugar, el modelo Mask R-CNN se optimiza mediante capas de fusión y deconvolución, lo que permite detectar características de pequeño tamaño, como poros y arrugas. En segundo lugar, se emplea un SR-GAN para aumentar la resolución de las imágenes de baja calidad, mejorando la precisión en la detección de problemas pequeños. Tercero, se entrenan modelos de segmentación específicos para cada tipo de problema, lo que optimiza la detección al reducir las interferencias de clases no relacionadas. La cuarta táctica consiste en utilizar modelos de segmentación específicos para cada dirección facial (frontal, lateral izquierda y derecha), ya que la posición y visibilidad de ciertos problemas varía según la orientación del rostro. Finalmente, la quinta táctica emplea un modelo de detección de landmarks faciales para descartar segmentaciones en áreas no faciales, como ojos, cejas y cabello, evitando falsos positivos.

Los resultados experimentales muestran que estas tácticas incrementan el rendimiento diagnóstico en un 32.58\% respecto a los modelos CNN convencionales, alcanzando una precisión de 83.38\%. Este enfoque no solo es preciso, sino que es adecuado para implementarse en dispositivos de bajo costo y en aplicaciones móviles, proporcionando una alternativa económica a las visitas clínicas. Los autores sugieren que este sistema podría ser de utilidad en clínicas de cuidado de la piel o como una herramienta accesible para el diagnóstico domiciliario.

\begin{figure}[h]
    \centering
    \includegraphics[width=0.5\textwidth]{2/figures/resultados de cuarto anteedente.png}
    \caption{Comparación del modelo propuesto con otros 6 modelos de redes neuronales}
    \label{fig:framework}
\end{figure}

%% Quinto antecedente: 



