%%%%%%%%
% \subsection{Segmentación de Imágenes}
% La segmentación de imágenes es el proceso de dividir una imagen en diferentes partes o regiones, con el objetivo de simplificar la representación de la imagen y hacerla más significativa y fácil de analizar. Este proceso es fundamental en aplicaciones de visión por computadora, especialmente en la detección de características morfológicas de la piel, como arrugas, poros y manchas \parencite{autor2020segmentacion}.

\subsection{Características Morfológicas de la Piel}
Las características morfológicas de la piel desempeñan un papel crucial en la evaluación de la salud y la estética facial, ya que ofrecen información valiosa sobre el estado general de la piel y sus posibles alteraciones. En este estudio, se consideran tres características clave: arrugas, poros y manchas. La correcta segmentación de estas características en imágenes faciales permite no solo el análisis cuantitativo de las mismas, sino también su monitoreo a lo largo del tiempo, contribuyendo al diseño de tratamientos cosméticos personalizados y a la evaluación de su efectividad.

\subsubsection{Arrugas}
Las arrugas son pliegues o líneas visibles en la superficie de la piel que se forman debido a la disminución de la elasticidad y el colágeno con el envejecimiento. Factores externos, como la exposición prolongada al sol, la contaminación y el tabaquismo, también contribuyen significativamente a su aparición. Además, las expresiones faciales repetitivas y la deshidratación de la piel pueden acelerar su desarrollo.

Desde el punto de vista estético, las arrugas se asocian con el envejecimiento y son una de las principales preocupaciones en el cuidado de la piel. Su segmentación precisa permite identificar su profundidad, longitud y densidad en diferentes áreas del rostro. Esta información es esencial para el desarrollo de productos antiarrugas y para evaluar la efectividad de tratamientos como cremas tópicas, terapias con láser o inyecciones de ácido hialurónico \cite{autor2021arrugas}.

\subsubsection{Poros}
Los poros son pequeñas aberturas en la piel a través de las cuales las glándulas sebáceas secretan sebo, un aceite natural que mantiene la piel hidratada y protegida. Su tamaño y visibilidad pueden variar según factores como la genética, el tipo de piel y los niveles hormonales. Los poros dilatados son una preocupación común, especialmente en personas con piel grasa, ya que pueden contribuir a la apariencia de una textura desigual y al desarrollo de imperfecciones, como puntos negros o acné.

La segmentación de poros en imágenes faciales proporciona una forma de cuantificar su tamaño, densidad y distribución, permitiendo una evaluación más objetiva. Esto es particularmente útil en estudios sobre tratamientos que buscan reducir su visibilidad, como peelings químicos, productos con retinoides o técnicas de microdermoabrasión \parencite{autor2020poros}.

\subsubsection{Manchas}
Las manchas son áreas de hiperpigmentación o hipopigmentación en la piel que resultan de una variedad de factores, incluyendo la exposición solar prolongada, cambios hormonales, envejecimiento y procesos inflamatorios. Ejemplos comunes incluyen el melasma, las manchas solares y las cicatrices post-inflamatorias.

Estas imperfecciones no solo afectan la apariencia de la piel, sino que también pueden indicar daño subyacente. Por ello, su detección y análisis temprano son fundamentales tanto para la prevención como para el tratamiento. La segmentación precisa de manchas en imágenes faciales permite identificar su forma, tamaño, color y evolución, lo que es útil para personalizar tratamientos como cremas despigmentantes, terapias con luz pulsada intensa (IPL) o procedimientos láser. Además, este análisis contribuye al diseño de cosméticos específicos que ayudan a unificar el tono de la piel \cite{autor2019manchas}.
%%%%%%%%%%
\subsubsection{Aplicaciones en Segmentación de Imágenes Faciales}  
En el ámbito del análisis de piel facial, las CNN son una herramienta fundamental para realizar segmentaciones precisas de características morfológicas, como arrugas, poros y manchas. Gracias a su capacidad para analizar imágenes a nivel de píxel, estas redes son capaces de identificar patrones y diferencias en la textura, el color y la estructura de la piel \parencite{autor2021deep}.

\paragraph{Segmentación de arrugas.}  
La segmentación de arrugas mediante CNN permite identificar líneas finas y pliegues en la piel, lo que es crucial para evaluar el envejecimiento facial y desarrollar tratamientos preventivos o correctivos. Este análisis automatizado es más preciso y rápido en comparación con las evaluaciones manuales, que pueden ser subjetivas y menos consistentes.

\paragraph{Detección de poros.}  
La identificación y segmentación de poros faciales es esencial para analizar problemas relacionados con la textura de la piel, como poros dilatados o acné. Las CNN pueden cuantificar el tamaño, la densidad y la distribución de los poros, facilitando la personalización de tratamientos según las necesidades específicas de cada individuo.

\paragraph{Segmentación de manchas.}  
Las manchas faciales, que pueden surgir debido a factores como la exposición solar o el envejecimiento, son una preocupación estética común. Las CNN permiten mapear su distribución y evaluar su progresión, ayudando a diagnosticar problemas como el melasma o el daño solar de manera temprana y objetiva \parencite{autor2020imagen}.

\subsubsection{Ventajas y Limitaciones}  
Las CNN ofrecen varias ventajas, entre ellas:
\begin{itemize}
    \item Alta precisión en la extracción y análisis de características complejas.
    \item Automatización de procesos que tradicionalmente dependen de evaluaciones subjetivas.
    \item Adaptabilidad a diferentes tipos de imágenes y tareas específicas.
\end{itemize}

Sin embargo, estas redes también presentan desafíos, como la necesidad de grandes volúmenes de datos etiquetados para el entrenamiento, el alto costo computacional y la posibilidad de sobreajuste si no se implementan técnicas adecuadas de regularización.

En el presente estudio, las CNN se utilizarán para desarrollar un sistema avanzado de segmentación de imágenes faciales, optimizado para la detección de arrugas, poros y manchas. Este enfoque busca contribuir al sector cosmético y de belleza, permitiendo una evaluación estética más precisa y la personalización de tratamientos cosméticos.


%%
%%%%%%%%%%

%%%
\subsection{Redes de Atención}  
Las redes de atención han surgido como una de las principales innovaciones en el campo de la segmentación de imágenes, permitiendo que los modelos se enfoquen de manera más eficiente en las regiones relevantes de una imagen. Este mecanismo resulta especialmente útil cuando se trabaja con imágenes complejas, como las faciales, donde ciertas áreas contienen características importantes para el análisis, pero pueden ser de menor tamaño o estar localizadas en posiciones no centrales.

\subsubsection{Mecanismo de Atención}  
El mecanismo de atención permite a la red asignar diferentes pesos a distintas partes de la imagen durante el proceso de segmentación. Este mecanismo es esencial para que el modelo pueda enfocarse en las áreas relevantes de la imagen, como arrugas, manchas o poros, sin perder detalles importantes de otras zonas. El concepto básico detrás de la atención es que no todas las partes de la imagen son igualmente importantes para la tarea en cuestión. Por lo tanto, la atención ayuda a los modelos a priorizar las áreas clave que impactan en el resultado final.  
\begin{itemize}
    \item \textbf{Funcionamiento:} La atención se puede aplicar de manera global o local. En el caso de la segmentación de imágenes faciales, por ejemplo, el modelo puede aprender a dar más importancia a las áreas alrededor de los ojos o la frente, donde las arrugas suelen ser más notorias.
    \item \textbf{Beneficio:} Aumenta la precisión de la segmentación al concentrarse solo en las características más relevantes y minimizar el "ruido" de otras regiones no significativas \parencite{autor2021atencion}.
\end{itemize}

\subsubsection{Atención Espacial}  
La atención espacial es un tipo específico de atención que asigna pesos según la ubicación espacial de los elementos dentro de la imagen. Esto permite al modelo resaltar áreas específicas de la imagen que contienen características clave para la segmentación.  
\begin{itemize}
    \item \textbf{Funcionamiento:} A través de la atención espacial, el modelo puede identificar patrones en las posiciones de las características de interés, como las arrugas en la zona de la frente o las manchas en la mejilla. Este enfoque es útil cuando las características relevantes están distribuidas de manera no uniforme en la imagen.
    \item \textbf{Aplicación:} Es particularmente eficaz en la segmentación de imágenes faciales, donde las características que deben segmentarse no están siempre en el mismo lugar de la imagen y varían según la persona y la expresión facial \parencite{autor2020spa}.
\end{itemize}

\subsubsection{Atención de Canal}  
La atención de canal se enfoca en las características dentro de los canales de la imagen, es decir, en las distintas representaciones de las características de la imagen que corresponden a las diferentes profundidades o colores de los filtros en una red convolucional. Este tipo de atención permite que la red enfoque su procesamiento en los canales que contienen la información más relevante para la tarea.  
\begin{itemize}
    \item \textbf{Funcionamiento:} En lugar de distribuir el enfoque en toda la imagen, la atención de canal resalta los canales específicos que contienen detalles cruciales, como la textura de la piel o las sombras que definen arrugas o manchas.
    \item \textbf{Beneficio:} Mejora la capacidad del modelo para diferenciar entre características de diferentes intensidades o patrones, lo cual es esencial en imágenes donde la variabilidad de la textura de la piel puede ser un desafío \parencite{autor2019canal}.
\end{itemize}

\subsubsection{Beneficios de las Redes de Atención}  
Las redes de atención proporcionan varios beneficios clave que mejoran la precisión y eficiencia de los modelos de segmentación, especialmente cuando se trabaja con imágenes complejas como las faciales:
\begin{itemize}
    \item \textbf{Precisión mejorada:} Al permitir que el modelo se enfoque en las regiones más importantes de la imagen, las redes de atención ayudan a obtener segmentaciones más precisas y detalladas.
    \item \textbf{Eficiencia en el procesamiento:} Reduciendo el "ruido" o la información irrelevante, las redes de atención aumentan la eficiencia computacional, ya que el modelo dedica más recursos a las áreas clave.
    \item \textbf{Versatilidad:} Las redes de atención se pueden combinar con otros modelos de segmentación, como U-Net o Mask R-CNN, para mejorar aún más la capacidad del modelo para realizar segmentaciones de alta calidad \parencite{autor2021beneficios}.
\end{itemize}

\subsubsection{Ejemplos de Modelos con Atención}  
Existen varios modelos que implementan mecanismos de atención para mejorar la segmentación de imágenes. Algunos ejemplos notables incluyen:
\begin{itemize}
    \item \textbf{Transformer:} El modelo Transformer ha sido exitoso en tareas de procesamiento de secuencias y también ha demostrado ser útil para segmentación de imágenes, particularmente al aplicar atención a nivel global en la imagen. Este modelo asigna pesos no solo localmente, sino también a nivel global, lo que mejora la precisión en tareas complejas de segmentación \parencite{autor2022transformer}.
    \item \textbf{SENet:} SENet es una arquitectura que utiliza un módulo de atención de canal para asignar diferentes importancias a los canales de características en una red convolucional. Este modelo ha demostrado ser eficaz en tareas de clasificación y segmentación, particularmente en la mejora de la detección de características sutiles, como pequeñas arrugas o manchas \parencite{autor2022senet}.
\end{itemize}




\begin{comment}

\subsection{Ecografía y las imágenes de ultrasonido}
Según \cite{pr_herrera2017diseimp}, la ecografía, que es una técnica de diagnóstico en donde se usan imágenes generadas por ultrasonido, es comúnmente desarrollado en las áreas de cardiología, ginecología, y otras más relacionadas. La popularidad de esta técnica se basa en la capacidad de las imágenes de alta calidad que se obtienen de este proceso, además de no ser un método invasivo o de radiación como muchos otros de su tipo.

Los sistema encargados de extraer las imágenes de ultrasonido son compuestos de distintas sensores que generan ondas de sonido para posteriormente analizar la respuesta de la interacción física con el campo de interés. Estas señales recibidas de regreso son digitalizados por una parte electrónica delantera que también transforman estos datos crudos en la imagen final. El funcionamiento de este proceso depende de la configuración de los sensores, el método usado para obtener las imágenes y las características de área de interés. \parencite{pr_camacho2022ultrasonicimg}

Algunas imágenes de ultrasonido de nódulos tiroideos se muestran en la Figura \ref{2:fig210}.

\begin{figure}[H]
	\begin{center}
		\includegraphics[width=0.40\textwidth]{2/figures/imagenes_ultrasonido_originales.png}
		\caption[Imágenes de ultrasonido de nódulos tiroideos]{Imágenes de ultrasonido de nódulos tiroideos. \\
		Fuente: \cite{pr_JERBI2023autoclassViTGAN}. \textit{Automatic classification of ultrasound thyroids images using vision transformers and generative adversarial networks}.}
		\label{2:fig210}
	\end{center}
\end{figure}


\subsection{Transfer Learning}
\cite{bk_geron2022handml} nos menciona que el Transfer Learning o Transferencia de Aprendizaje es un técnica usada en el campo de Deep Learning que permite el uso de algunas capas de un modelo ya definido y entrenado previamente en un nuevo modelo que necesite ser entrenado en una tarea similar al que se desarrolla el modelo original. Las capas destinadas al reuso son normalmente las más cercanas a la entrada o también conocidas como capas inferiores. El beneficio de usar esta técnica radica en dos puntos importantes: cantidad de datos requeridos y velocidad de entrenamiento del modelo; es decir, la cantidad de datos que se deben usar para entrenar un modelo de alto desempeño se reduce considerablemente, mientras que el tiempo requerido para terminar este proceso es menor comparándolo a si lo entrenaran desde cero.

Para que esta técnica funcione debidamente, las capas más cercanas a la salida, conocidas también como capas de alto nivel, deben ser reemplazadas, esto debido a que son más específicas de las tareas del modelo original. Esto también incluye a la capa final, ya que posiblemente no tenga la cantidad de salidas necesarias para completar satisfactoriamente la nueva tarea.

En la Figura \ref{2:fig211} se presenta de forma gráfica la técnica.

\begin{figure}[H]
	\begin{center}
		\includegraphics[width=0.70\textwidth]{2/figures/transfer_learning.PNG}
		\caption[Ejemplo de Transfer Learning]{Ejemplo de Transfer Learning. \\
		Fuente: \cite{bk_geron2022handml}. \textit{Hands-on machine learning with Scikit-Learn, Keras, and TensorFlow}.}
		\label{2:fig211}
	\end{center}
\end{figure}


\subsection{Data Augmentation}

Según \cite{bk_geron2022handml} el Aumento de Datos o Data Augmentation es una técnica de regularización que permite reforzar la cantidad de muestras en un conjunto de datos. Esto se realiza a través de la generación de nuevas instancias similares a los originales; es decir, las personas no deberían ser capaces de diferenciar una imagen generada de una del propio conjunto de datos.

Para generar estas nuevas muestras, normalmente se aplican diferentes transformaciones a las instancias del conjunto de datos original. Estas transformaciones pueden ser; por ejemplo, una simple rotación o recorte de la imagen, siempre y cuando no altere por completo su sentido como es el caso de voltear una imagen de texto de forma horizontal. 

El principal beneficio de esta técnica es que permite reducir el sobreajuste de los modelos entrenados. 

En la Figura \ref{2:fig212} se muestran algunas transformaciones que se pueden hacer al aplicar el Aumento de Datos. 

\begin{figure}[H]
	\begin{center}
		\includegraphics[width=0.85\textwidth]{2/figures/data_aug.PNG}
		\caption[Ejemplo de Data Augmentation]{Ejemplo de Data Augmentation. \\
		Fuente: \cite{bk_geron2022handml}. \textit{Hands-on machine learning with Scikit-Learn, Keras, and TensorFlow}.}
		\label{2:fig212}
	\end{center}
\end{figure}

\end{comment}