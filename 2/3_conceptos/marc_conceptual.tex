
\subsection{Redes Neuronales Convolucionales (CNN)}

Las redes neuronales convolucionales constituyen una arquitectura fundamental dentro del aprendizaje profundo, diseñada específicamente para procesar datos con una estructura en forma de rejilla, como las imágenes. Estas redes aplican operaciones convolucionales para detectar patrones espaciales jerárquicos, desde características locales simples hasta representaciones complejas de alto nivel, lo que las hace altamente efectivas en tareas como clasificación, detección y segmentación de imágenes. Gracias a su capacidad de generalización y aprendizaje automático supervisado, las CNN se han convertido en herramientas esenciales en aplicaciones biomédicas, donde permiten analizar imágenes faciales para identificar rasgos morfológicos sutiles como arrugas o manchas, facilitando así diagnósticos más objetivos y eficientes \parencite{lecun1998gradient, krizhevsky2012, ronneberger2015}.

\subsection{Segmentación de Imágenes}
La segmentación de imágenes es una técnica computacional que tiene como objetivo dividir una imagen en múltiples regiones coherentes, cada una de las cuales representa una estructura significativa dentro de la escena. Este proceso permite una interpretación más estructurada y manejable de los datos visuales, sirviendo como paso previo esencial en tareas de reconocimiento y análisis morfológico. En contextos médicos y cosméticos, la segmentación resulta crucial para identificar con precisión características como arrugas o manchas, lo cual contribuye a evaluaciones estéticas personalizadas y al seguimiento de intervenciones terapéuticas \parencite{autor2020segmentacion}.

\subsection{Arrugas}

Las arrugas son marcas visibles en la superficie cutánea que emergen como parte del proceso natural de envejecimiento, pero también pueden intensificarse por factores ambientales y hábitos de vida. Su aparición se asocia principalmente con la pérdida progresiva de colágeno y elastina, proteínas que otorgan firmeza y elasticidad a la piel. A medida que disminuye la regeneración celular, la piel se vuelve más fina y propensa a formar pliegues permanentes, siendo estos signos clave tanto en la evaluación estética como en estudios dermatológicos de envejecimiento \parencite{autor2021arrugas}.

\subsection{Manchas}
Las manchas en la piel representan alteraciones en la pigmentación que se manifiestan como zonas hiperpigmentadas o hipopigmentadas, derivadas de causas intrínsecas y extrínsecas como exposición prolongada al sol, envejecimiento, trastornos hormonales o procesos inflamatorios. Estas manifestaciones, que incluyen el melasma, lentigos solares o hiperpigmentaciones postinflamatorias, constituyen una de las principales preocupaciones estéticas, ya que afectan el tono uniforme del rostro. Su evaluación objetiva mediante imágenes ha cobrado gran relevancia en dermatología estética \parencite{autor2019manchas}.

\subsection{Segmentación de arrugas}  
La segmentación automática de arrugas permite la identificación precisa de líneas y pliegues en la piel facial, representando una herramienta poderosa para estudios de envejecimiento y tratamientos antienvejecimiento. Empleando CNN, este proceso automatiza la detección de patrones morfológicos a partir de imágenes, superando las limitaciones de las evaluaciones subjetivas. Esta metodología facilita el análisis cuantitativo de la progresión del envejecimiento cutáneo y la eficacia de intervenciones cosméticas \parencite{autor2020segmentacion}.

\subsection{Segmentación de manchas}  
La segmentación de manchas cutáneas mediante redes neuronales profundas ofrece un enfoque sistemático para detectar y delimitar zonas pigmentadas con alta precisión. Esta técnica no solo permite la localización de manchas visibles, sino que también facilita la cuantificación y monitoreo de su evolución, lo cual es crucial en tratamientos dermatológicos. Las CNN han demostrado ser especialmente eficaces en esta tarea debido a su habilidad para aprender representaciones complejas a partir de datos etiquetados \parencite{autor2020segmentacion}.

\subsection{U-Net:} 
La arquitectura U-Net ha sido diseñada para tareas de segmentación de imágenes biomédicas, destacándose por su estructura tipo encoder-decoder con conexiones de salto que preservan la información espacial detallada. Esta disposición permite que el modelo capture tanto las características contextuales como los bordes precisos de los objetos segmentados, siendo especialmente útil en aplicaciones donde la delimitación morfológica exacta es crítica, como en imágenes dermatológicas o radiológicas \parencite{ronneberger2015}.

\subsection{Attention U-Net:} 
Attention U-Net es una extensión de la arquitectura U-Net que incorpora mecanismos de atención, los cuales permiten al modelo enfocar su capacidad de representación en regiones morfológicamente relevantes de la imagen. Esta mejora se traduce en un desempeño superior en escenarios donde las estructuras a segmentar son pequeñas o están parcialmente ocultas, al priorizar aquellas zonas con mayor información semántica durante el proceso de decodificación \parencite{oktay2018attentionunet}.

\subsection{U-Net con codificador MiT-B0:} 
La integración del codificador MiT-B0, basado en transformadores, dentro de la arquitectura U-Net representa una innovación que combina la capacidad de los transformadores para capturar relaciones globales con la eficiencia estructural de U-Net. Esta variante permite una mejor generalización en tareas de segmentación complejas, donde es necesario analizar contextos visuales amplios manteniendo precisión local, como ocurre en imágenes dermatológicas con estructuras difusas \parencite{Yeom2025}.

\subsection{Mask R-CNN} 
Mask R-CNN es una arquitectura avanzada de segmentación de instancias que extiende Faster R-CNN mediante la incorporación de una rama adicional encargada de predecir máscaras a nivel de píxel para cada objeto detectado. Esta combinación de detección, clasificación y segmentación permite resultados precisos en imágenes con múltiples objetos o características faciales complejas. Su versatilidad ha sido ampliamente demostrada en contextos como el análisis médico, el reconocimiento facial y la visión computacional aplicada \parencite{Jung2019AutomaticNucleiSegmentation}.

