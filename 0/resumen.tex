%La línea de abajo es para quitar encabezado
\thispagestyle{plain}

\chapter*{Resumen}
\markboth{Resumen}{Resumen}
\addcontentsline{toc}{chapter}{Resumen}

El presente trabajo de investigación tiene como objetivo desarrollar un sistema de segmentación automática de características morfológicas faciales, específicamente arrugas y manchas, empleando redes neuronales convolucionales (CNN). La motivación principal radica en el creciente interés de la industria cosmética por ofrecer diagnósticos estéticos personalizados y en la necesidad médica de detectar de manera temprana anomalías cutáneas que pueden estar asociadas con enfermedades degenerativas o cáncer de piel. Para ello, se implementaron y compararon distintas arquitecturas de segmentación, entre ellas U-Net, U-Net con atención, Mask R-CNN y U-Net con codificador MiT-B0, utilizando métricas como entropía cruzada, precisión, coeficiente de Sørensen–Dice e índice de Jaccard. Los resultados evidencian que el modelo U-Net con atención obtuvo el mejor rendimiento, con una precisión promedio de 0.911 y un Dice Score de 0.810. Posteriormente, se integró este modelo en una aplicación web funcional que permite la carga e inferencia de imágenes faciales en tiempo real, proporcionando segmentaciones visuales con codificación por color para facilitar su interpretación. Esta herramienta representa un avance significativo en la automatización del análisis estético y clínico de la piel, con potencial de aplicación en entornos dermatológicos y de cosmética avanzada. Se concluye que el uso de CNN, especialmente con mecanismos de atención, es altamente efectivo para la segmentación morfológica facial, y se proponen líneas futuras de trabajo enfocadas en la mejora del dataset, optimización del modelo y validación clínica del sistema desarrollado.

\textbf{Palabras clave:} Segmentación, redes neuronales convolucionales, arrugas, manchas, imagen de la piel, cosmética, tratamiento personalizado.

\clearpage
\chapter*{Abstract}
\markboth{Abstract}{Abstract}

The objective of this research work is to develop an automatic segmentation system for facial morphological features, specifically wrinkles and blemishes, using convolutional neural networks (CNNs). The main motivation lies in the growing interest of the cosmetics industry in offering personalized aesthetic diagnoses and the medical need for early detection of skin abnormalities that may be associated with degenerative diseases or skin cancer. To this end, different segmentation architectures were implemented and compared, including U-Net, U-Net with attention, Mask R-CNN, and U-Net with MiT-B0 encoder, using metrics such as cross entropy, accuracy, Sørensen–Dice coefficient, and Jaccard index. The results show that the U-Net with attention model obtained the best performance, with an average accuracy of 0.911 and a Dice Score of 0.810. Subsequently, this model was integrated into a functional web application that allows the loading and inference of facial images in real time, providing visual segmentations with color coding for easy interpretation. This tool represents a significant advance in the automation of aesthetic and clinical skin analysis, with potential for application in dermatological and advanced cosmetics settings. It is concluded that the use of CNNs, especially with attention mechanisms, is highly effective for facial morphological segmentation, and future lines of work are proposed, focusing on dataset improvement, model optimization, and clinical validation of the developed system.

\textbf{Keywords:} Segmentation, convolutional neural networks, wrinkles, spots, skin image, cosmetics, personalized treatment.
