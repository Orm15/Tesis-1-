%La línea de abajo es para quitar encabezado
\thispagestyle{plain}

\chapter*{Resumen}
\markboth{Resumen}{Resumen}
\addcontentsline{toc}{chapter}{Resumen}

La industria cosmética y de cuidado de la piel ha experimentado un crecimiento acelerado en las últimas décadas. Sin embargo, a pesar de la diversidad de productos y tratamientos disponibles, persisten desafíos importantes en la evaluación precisa y personalizada de problemas estéticos, como arrugas, poros dilatados y manchas faciales. Uno de los mayores obstáculos es la falta de herramientas tecnológicas avanzadas que permitan un análisis profundo y objetivo de las características morfológicas de la piel.

Este trabajo de investigación busca desarrollar un sistema de segmentación basado en redes neuronales convolucionales (CNN) para detectar y evaluar características morfológicas clave de la piel facial, como arrugas, poros y manchas. El sistema propuesto mejorará la precisión en la evaluación estética y la personalización de tratamientos cosméticos. Se utilizarán arquitecturas de CNN adaptadas para segmentar con precisión las imágenes de la piel facial, basándose en un conjunto de datos representativo que permita entrenar y validar el modelo. Las métricas de desempeño, como la precisión, el recall y el F1-score, serán empleadas para evaluar la eficiencia del sistema.

Este avance contribuirá tanto al ámbito de la dermatología computacional como al de la cosmética, optimizando los procesos de diagnóstico y recomendación de productos personalizados. La implementación de este sistema permitirá mejorar la precisión de los diagnósticos y personalizar tratamientos de manera más efectiva, beneficiando a consumidores que buscan soluciones específicas para sus necesidades estéticas.

\textbf{Palabras clave:} Segmentación, redes neuronales convolucionales, arrugas, poros, manchas, imagen de la piel, cosmética, tratamiento personalizado.

\clearpage
\chapter*{Abstract}
\markboth{Abstract}{Abstract}

The cosmetics and skincare industry has seen rapid growth in recent decades. However, despite the variety of products and treatments available, there remain significant challenges in the precise and personalized evaluation of aesthetic issues, such as wrinkles, enlarged pores, and skin spots. One of the major obstacles is the lack of advanced technological tools that allow for a deep and objective analysis of the skin's morphological features.

This research aims to develop a segmentation system based on Convolutional Neural Networks (CNN) to detect and evaluate key morphological features of facial skin, such as wrinkles, pores, and spots. The proposed system will enhance the accuracy of aesthetic evaluations and the personalization of cosmetic treatments. CNN architectures will be used to segment facial skin images accurately, using a representative dataset to train and validate the model. Performance metrics such as accuracy, recall, and F1-score will be used to assess the system's efficiency.

This advancement will contribute to both computational dermatology and cosmetics by optimizing diagnostic processes and personalized product recommendations. The implementation of this system will improve diagnostic precision and allow for more effective treatment personalization, benefiting consumers seeking solutions tailored to their specific aesthetic needs.

\textbf{Keywords:} Segmentation, convolutional neural networks, wrinkles, pores, spots, skin image, cosmetics, personalized treatment.
