%La línea de abajo es para quitar encabezado
%\thispagestyle{plain}

\chapter*{Introducción}
\markboth{Introducción}{Introducción}
\addcontentsline{toc}{chapter}{Introducción}
En las últimas décadas, el análisis automático de imágenes faciales ha cobrado una relevancia creciente en campos como la medicina, la cosmética, la biometría y la inteligencia artificial. En particular, la detección de características morfológicas de la piel como arrugas y manchas no solo representa un interés estético dentro de la industria cosmética, sino que también puede ofrecer indicios clínicos relevantes sobre el envejecimiento cutáneo, enfermedades dermatológicas e incluso condiciones degenerativas o precancerosas. Este contexto ha impulsado la búsqueda de soluciones tecnológicas que permitan automatizar la segmentación y análisis de dichas características faciales, haciendo uso de técnicas avanzadas de visión por computadora y aprendizaje profundo.

Entre los enfoques más efectivos en tareas de segmentación de imágenes se encuentran las redes neuronales convolucionales (CNN), las cuales han demostrado resultados superiores frente a métodos tradicionales, debido a su capacidad para aprender patrones espaciales jerárquicos de manera automática y eficiente. Modelos como U-Net y sus variantes han sido ampliamente utilizados en la segmentación médica y biométrica por su arquitectura tipo encoder-decoder, que permite preservar información espacial fina mientras se capturan representaciones abstractas de alto nivel.

En este trabajo se propone un sistema que emplea una arquitectura CNN, específicamente U-Net con mecanismos de atención, para segmentar arrugas y manchas en imágenes faciales. Con el objetivo de evaluar el desempeño del modelo, se desarrolló una metodología experimental que incluye la construcción de un dataset multiclase, el preprocesamiento de imágenes con técnicas de aumentación de datos, y la comparación de distintas arquitecturas de segmentación. Las métricas utilizadas incluyeron precisión, pérdida por entropía cruzada, coeficiente de Sørensen–Dice e índice de Jaccard, seleccionadas por su capacidad de reflejar el solapamiento entre predicción y verdad de terreno en tareas de segmentación densa.

Como resultado, se construyó una aplicación web funcional que permite a los usuarios cargar imágenes y recibir segmentaciones visuales codificadas por colores, lo que facilita la interpretación de las regiones correspondientes a arrugas, manchas y fondo. Esta herramienta no solo representa un aporte al análisis automatizado de la piel, sino que también abre la posibilidad de futuras aplicaciones clínicas y cosméticas para diagnóstico no invasivo, seguimiento de tratamientos y desarrollo de productos personalizados.

La presente tesis se estructura en capítulos que abordan la revisión del estado del arte, el planteamiento metodológico, el diseño del sistema, los experimentos realizados, los resultados obtenidos y las conclusiones derivadas del estudio, así como recomendaciones para futuras investigaciones.