%La línea de abajo es para quitar encabezado
%\thispagestyle{plain}

\chapter*{Introducción}
\markboth{Introducción}{Introducción}
\addcontentsline{toc}{chapter}{Introducción}
En resumen, el principal objetivo de la presente investigación es el siguiente: desarrollar un sistema avanzado de segmentación de características morfológicas de la piel facial, es decir, las arrugas y las manchas, utilizando redes neuronales convolucionales (CNN). En otras palabras, este sistema tiene como objetivo mejorar la precisión de la valoración estética y, por lo tanto, fortalecer la personalización del tratamiento cosmético y favorecer el desarrollo de la industria de la piel.
El sector cosmético se ha desarrollado de forma sostenida en las últimas décadas, a medida de una creciente preocupación en torno a problemas estéticos y el interés de mantener la piel sana y joven. No obstante, la mayor parte del diagnóstico de enfermedades cutáneas se basa en la observación manual o utilizando herramientas tradicionales, que son a menudo inexactas y no objetivas. Por lo tanto, se necesita urgentemente el desarrollo de nuevas tecnologías para evaluar con mayor detalle y precisión las características de la piel.
En este contexto, las CNN han demostrado ser altamente efectivas para tareas de segmentación y análisis de imágenes, permitiendo identificar patrones complejos en datos visuales. Este estudio se enfoca en aplicar estas técnicas para desarrollar un modelo capaz de detectar y segmentar características clave de la piel, como arrugas y manchas faciales, con el fin de brindar un análisis objetivo y detallado.
La investigación sigue un enfoque metodológico que incluye la recopilación de datos, el desarrollo y prueba de modelos basados en CNN, y la evaluación de su desempeño mediante métricas como precisión, recall, F1-score y AUC-ROC. Este enfoque asegura la validación y confiabilidad del sistema desarrollado, destacando su aplicabilidad en el sector cosmético.