\chapter{Conclusiones y Recomendaciones}
\section{Conclusiones}
Una vez definido el problema central relacionado con la segmentación automatizada de deformaciones morfológicas faciales y establecido el marco de objetivos e hipótesis, se logró cumplir con el propósito general de esta investigación: diseñar un sistema capaz de segmentar arrugas y manchas en imágenes faciales mediante Redes Neuronales Convolucionales. Este objetivo se abordó considerando como criterio principal el uso exclusivo de imágenes faciales humanas y, como enfoque metodológico, la implementación de un modelo de aprendizaje profundo basado en redes neuronales convolucionales, específicamente una arquitectura U-Net con mecanismos de atención.

En relación con los objetivos específicos, se evidenció que el análisis de modelos propuestos en investigaciones previas tuvo un impacto significativo en la selección de características relevantes y en la construcción del enfoque metodológico adoptado. Esta revisión permitió validar hipótesis provenientes de la literatura sobre el desempeño superior de arquitecturas basadas en redes neuronales convolucionales frente a enfoques tradicionales de aprendizaje automático, especialmente en tareas de segmentación de imágenes. Los resultados obtenidos, como se ven el Tabla \ref{tab:result_models}, durante las fases de evaluación cuantitativa respaldan esta elección, demostrando mejoras sustanciales en métricas como loss, precisión, Dice Score e IoU.

A partir de los resultados mostrados en la Tabla \ref{tab:result_models}, se concluye que el modelo U-Net con mecanismos de atención obtuvo el mejor desempeño global en la tarea de segmentación de características morfológicas faciales, superando significativamente a sus variantes y a otras arquitecturas evaluadas. Este modelo logró una pérdida (entropía cruzada) de 0.1158 y una precisión de 0.911, posicionándose por encima del U-Net estándar y de U-Net con codificador MiT-B0, así como muy por delante de Mask R-CNN, que mostró un rendimiento considerablemente inferior. Las métricas del índice de Sørensen–Dice (0.810) y el coeficiente de Jaccard (0.852) reflejan la eficacia del modelo propuesto en términos de solapamiento con las segmentaciones reales. Sin embargo, también se identificó que el rendimiento de algunos modelos estuvo condicionado por las características específicas de su diseño. Por ejemplo, el modelo con codificador MiT-B0, pese a incorporar componentes basados en transformadores, presentó dificultades para generalizar, lo que se evidenció en una baja precisión (0.443) y un Dice Score de apenas 0.320. Este comportamiento sugiere que, si bien la arquitectura es prometedora, su desempeño podría estar limitado por el tamaño del conjunto de datos o la sensibilidad a variaciones de iluminación y contraste presentes en las imágenes faciales. Estas observaciones subrayan la importancia de una adecuada selección arquitectónica y del ajuste fino de hiperparámetros en función de las características del problema. En este caso, la integración de mecanismos de atención demostró ser clave para mejorar el enfoque espacial del modelo, potenciando su capacidad para segmentar con mayor precisión regiones como arrugas y manchas en entornos clínicos o cosméticos.

A nivel individual, la aplicación web desarrollada para la segmentación de características morfológicas faciales demostró un comportamiento consistente en todas las métricas de evaluación aplicadas, desde las más convencionales como la precisión global, hasta aquellas más sensibles al desbalance de clases, como el índice de Sørensen–Dice y el coeficiente de Jaccard. El modelo U-Net con atención, integrado en el sistema, fue entrenado sin mayores dificultades y mostró una alta estabilidad durante el proceso, reflejando una baja pérdida de entrenamiento y validación. La arquitectura implementada mantuvo una adecuada capacidad de generalización a pesar de ciertas limitaciones inherentes al conjunto de datos, como variabilidad en la iluminación o diversidad en los tonos de piel. En este sentido, el entorno de inferencia fue capaz de gestionar adecuadamente estos desafíos y generar segmentaciones robustas en la mayoría de los casos evaluados. Esta fiabilidad es especialmente relevante considerando que la aplicación web fue diseñada para operar en tiempo real y con imágenes proporcionadas directamente por el usuario, lo que introduce una mayor heterogeneidad en las entradas. Por último, sí se evidenció que la calidad y resolución de las imágenes puede impactar en la nitidez de las regiones segmentadas, especialmente en manchas de baja intensidad o arrugas poco marcadas. No obstante, el sistema mantuvo métricas altas y estables en todos los ensayos realizados, consolidándose como una herramienta eficaz y funcional tanto para aplicaciones clínicas como cosméticas.

\section{Recomendaciones}

Para mejorar la capacidad de generalización del modelo, se recomienda incrementar el volumen del conjunto de datos, incorporando imágenes faciales con mayor variabilidad en condiciones de iluminación, tono y tipo de piel, presencia de maquillaje y expresiones faciales diversas. La inclusión de muestras procedentes de bases de datos clínicas también podría enriquecer la robustez del sistema frente a casos dermatológicos reales.

Aunque la aplicación fue desarrollada para su uso en entorno web local, se sugiere su adaptación para su ejecución en dispositivos móviles o entornos con recursos computacionales restringidos. Para ello, se podrían explorar técnicas de cuantización o poda de redes, así como la conversión del modelo a formatos ligeros como ONNX o TensorFlow Lite.

Con el fin de mejorar la experiencia del usuario y validar los resultados obtenidos, se propone incorporar un módulo de evaluación automática que permita comparar las máscaras generadas por el modelo con etiquetas reales, en caso de disponer de ellas, o bien aplicar mecanismos de retroalimentación por parte del usuario para ajustar la sensibilidad del modelo.

Si bien la U-Net con atención demostró un excelente rendimiento, futuras investigaciones podrían considerar arquitecturas más recientes como U-Net++, DeepLabV3+, o segmentadores basados en transformadores (p. ej., SegFormer), los cuales han mostrado resultados prometedores en tareas de segmentación médica y facial.

Se recomienda realizar experimentos adicionales de ajuste fino de hiperparámetros como el tamaño del batch, la tasa de aprendizaje y los pesos de clase en la función de pérdida, así como aplicar técnicas como dropout o data augmentation adaptativa, con el fin de minimizar el riesgo de sobreajuste en datasets más extensos.

